% otftotfm -a -e ec -fkern FRAMDCN.TTF FGothicCond
\input eplain
\pdfpageheight=11in
\pdfpagewidth=8.5in
\bottommargin=0.8in
\frenchspacing
\def\sp#1{^{(#1)}}
\parskip=0pt
\parindent=2em
\overfullrule=0pt
\def\prepic{\vskip-\baselineskip}
\def\prepicsm{\vskip-8pt\relax}
\def\prepiclg{\vskip-15pt\relax}
\def\pic#1#2{\pdfximage width #1 {#2}\pdfrefximage\pdflastximage}
\def\picbox#1#2{\hbox{\pic{#1}{#2}}}
\def\pics#1#2{\hbox{\hskip -.15in \pic{1.7in}{#1stdev1#2.png}\pic{1.7in}{#1stdev2#2.png}\pic{1.7in}{#1stdev4#2.png}\pic{1.7in}{#1nonsep#2.png}}}
\def\caption#1#2{\vtop{\noindent\ss{\ssbf #1} -- #2\hfil}}
\def\todo#1{{\bf TODO: #1}}
\def\hr{\vskip 2pt \hrule \vskip 2pt}
\def\tag#1{\noindent\vtop to 0pt{\llap{\sl (#1)\quad}\vss}\ignorespaces}
\def\hbarchart#1#2{\hbox{\relax
\hbox to 1cm{\hss\vrule height 8pt width #2cm}\relax
\vrule width .2pt height 10pt depth 2pt \relax
\hbox to 1cm{\vrule height 8pt width #1cm\hss}}}
\def\valbar#1{#1&\hbarchart{#1}{-#1}}
\font\ss=FranklinGothicBook at 10pt
\font\ssbf=FranklinGothicDemi at 10pt
\def\assignment#1{{\noindent \it #1\par}}
\def\sec#1{\vskip8pt\noindent{\bf #1}\par\penalty10000\vskip2pt\penalty10000\noindent\ignorespaces}
\def\*{\par\noindent\hangindent=2em\hangafter=1\hbox to 1em{\hfil}\hbox to 1em{$\bullet$\hfil}\ignorespaces}

\def\term#1{{\tt #1}}

\hbox to \hsize{\hfil\bf Predicting Loan Outcomes using Machine Learning --- CS109a Project Milestone 4\hfil}
\hbox to \hsize{\hfil David Modjeska and Andrew Greene\hfil}

\doublecolumns

\sec{Abstract}
In this milestone report, we build baseline models on the Lending Club (LC) dataset. We build on our work in the previous milestone, in which we evaluated the available columns.

\sec{Curating our data}
We begin by focusing on loans issued in the years 2011, 2012, and 2013. There are far fewer loans issued prior to 2011, and (possibly because of economic conditions at the time), they have a higher overall default rate. For loans issued in 2014 and later, there is a censoring effect because not all loans have had enough time to mature. We also exclude loans with a 60-month term for the same reason.

Next, we choose the predictor columns which we will use. They are listed in Fig.~1.

A few notes on these: months\_since\_earliest\_credit is the number of months between the issue\_date and the earliest\_credit. This is to correct for the fact that loans are issued at different times. The continuous economic predictors are data imported from the Federal Reserve's ``FRED'' database for the month identified as issue\_date. Since future loan predictions will not be able to use values of issue\_date found in our dataset, we assume that the relevant information latent in the issue\_date column can be represented by significant economic indicators at that time. 

Our preliminary exploration showed that employ\_length had a surprising property: it is presented as a categorical predictor, and although one could convert it to a continuous predictor, it was actually only the ``missing'' value that had a statistically significantly different mean default rate -- and since in our models continuous predictors use their median to impute missing values, if we were to convert this to a continuous predictor we would lose the one value that makes this predictor worth keeping!

As a quick sanity-check, we create a Random Forest and see what predictors it considers important. (See Fig.~2) Topping the list is annual\_income, which is reassuring, although the state-specific one-hot-encoded predictors follow, which is interesting. 

By comparison, a simple (balanced) Logistic Regression model gives the values in Fig.~3 as the coefficients with the largest and smallest (aboslute) values. These put far more emphasis on the (one-hot-encoded) employment title (which, we observe, contains a mix of (a) title-cased values, (b) lower-case values, and (c) company names). Strangely, ``Manager'' has a +0.92 coefficient, indicating a low risk of default, while ``manager'' has a -0.71 coefficient, indicating a high risk.

\sec{Baseline models}
The plausibility of the coefficients from the Logistic Regression model gives us confidence to go ahead with this data set and construct one exemplar of each family of models.

In the following table, we treat ``Paid in full'' as a positive (1) outcome; this lets us focus on the precision rather than the overall classification accuracy (CA). For the individual member of the Lending Club, there are sufficient loans available for funding that there is virtually no opportunity cost to passing up a loan that will be paid. It is much more important to avoid non-performing loans. (In Milestone 5 we discuss this in more depth.) Scores in the following table are on the test data.

\vskip 4pt
\halign{#\hfil\ &\hfil#\hfil\ &\hfil #\hfil\ &\hfil#\hfil\cr
\bf Model                  &\bf CA    &\bf Precision   &\bf F1\cr
\noalign{\vskip2pt \hrule\vskip -1pt}\cr
Always 1               &0.848    &0.848    &0.264 \cr
Unbalanced LogReg      &0.848    &0.849    &0.262 \cr
Balanced LogReg        &0.622    &0.902    &0.155 \cr
QDA                    &0.165    &0.863    &0.030 \cr
Random Forest          &0.848    &0.848    &0.264 \cr
Random Forest w/bias   &0.433    &0.901    &0.153 \cr
Balanced SVC           &0.848    &0.849    &0.263 \cr
Balanced SVC w/bias    &0.603    &0.904    &0.151 \cr
}
\vskip 4pt

Because the classes are imbalanced, the methods that do not compensate for this are unable to detect those loans that are likely to default, yielding both precision and overall classification accuracy of the 85\% of loans that are paid in full. Those other models that {\it do} compensate are more likely to identify loans as risky: both correctly and incorrectly. For those models, the overall classification accuracy (and F1 scores) drop, but the {\it precision} increases, which is what we want to see. (Fig.~4)

We do not consider KNN because of both the cardinality and the dimensionality of the data.

\sec{Next Steps}
Given these baselines, we will be able to compare the performance of the models we continue to develop. Our plans for the remainder of this project are discussed in Milestone 5.

\sec{Code}
The code for this project can be found on GitHub at {\tt https://github.com/amgreene/cs109a} It is a private repository but we are happy to share access to it.

\singlecolumn\vfill\eject
\headline{\rm Predicting Loan Outcomes --- Milestone \#4 \hfill David~Modjeska and Andrew~Greene}

\sec{Fig. 1: List of Predictors}
{\obeylines{\bf Continuous Predictors}\catcode`\_=12
{\tt annual_income
delinq_2_yrs 
desc_len
inquiry_6_mos
installment
loan_amount 
loan_term
months_since_earliest_credit
months_since_last_record
open_accounts
revol_util
total_accounts
unemploy}
\par\noindent{\bf Categorical Predictors}
{\tt address_state
employ_length
employ_title
home_owner
initial_list_status
loan_purpose }
\par\noindent{\bf Continuous Economic Predictors}
{\tt cpi 
dti 
gdp 
ipr 
rir}
}

\sec{Fig. 2: Importance in Random Forest}
\picbox{6in}{ms3-rf-importance.png}

\sec{Fig. 3: Coefficients in Logistic Regression}
{\obeylines\obeyspaces\tt\parindent=0pt\catcode`\_=12%
employ_title__Sales                 -1.481170
employ_title__Engineer              +1.453079
loan_amount                         -1.150217
installment                         +1.080646
employ_title__Paralegal             -1.048912
employ_title__Software_Engineer     +0.993126
employ_title__Kaiser_Permanente     +0.970401
employ_title__Branch_Manager        +0.949948
employ_title__Assistant_Manager     +0.934904
employ_title__Manager               +0.927453
employ_title__Attorney              +0.910822
address_state__IA                   -0.880930
employ_title__President             +0.834595
employ_title__truck_driver          -0.827184
employ_title__Wells_Fargo_Bank      +0.804132
employ_title__manager               -0.714929
employ_title__supervisor            -0.712204
employ_title__Walmart               -0.649203
employ_title__Executive_Assistant   +0.647786
employ_title__Operations_Manager    +0.647447
loan_purpose__small_business        -0.615268
employ_title__sales                 -0.594256
loan_purpose__credit_card           +0.547049
home_owner__OTHER                   -0.512765
employ_title__Sales_Manager         -0.498320
address_state__DC                   +0.494797
address_state__WY                   +0.484090
employ_title__Truck_Driver          +0.476451
employ_title__RN                    -0.474911
home_owner__NONE                    +0.458328
                                       ...   
desc_len                            +0.055595
employ_length__4                    +0.054544
address_state__NC                   +0.051968
employ_title__Program_Manager       +0.051700
address_state__MI                   -0.046741
initial_list_status__w              -0.045878
employ_title__IBM                   +0.037288
employ_length__3                    +0.034592
employ_length__5                    +0.034294
address_state__OK                   -0.032820
months_since_last_record            -0.031870
address_state__NM                   +0.030634
address_state__IL                   +0.029021
address_state__WI                   -0.027886
address_state__CA                   -0.027058
address_state__AK                   +0.026770
employ_length__10                   -0.025127
address_state__SC                   -0.025030
months_since_earliest_credit        +0.024779
home_owner__OWN                     -0.023421
employ_length__7                    -0.021551
address_state__AR                   -0.017962
employ_length__0                    -0.016399
address_state__OH                   -0.009843
address_state__CT                   -0.007454
address_state__KS                   -0.004893
address_state__GA                   +0.001055
address_state__ND                   -0.000000
address_state__MS                   -0.000000
loan_purpose__educational           -0.000000}

\sec{Fig. 4: Comparison of Results}
\picbox{6in}{ms4-results.png}


\bye
